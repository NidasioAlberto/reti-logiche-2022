\subsection{Sintesi}

Il dispositivo progettato è perfettamente sintetizzabile ed occupa le risorse mostrate in figura \ref{table:risorse}.

\begin{figure}[!ht]
    \centering
    \begin{tabular}{|c | c |}
        \hline
        Resource & Utilization \\
        \hline
        LUT      & $63$        \\
        \hline
        FF       & $47$        \\
        \hline
        IO       & $38$        \\
        \hline
        BUFG     & $1$         \\
        \hline
    \end{tabular}
    \caption{Risorse hardware post implementazione}
    \label{table:risorse}
\end{figure}

Analizzando il timing report mostrato in figura \ref{table:timingreport} è possibile vedere che il tempo di esecuzione del componente rispetta il limite di $100ns$ per il periodo di clock.

\begin{figure}[!ht]
    \centering
    \begin{tabular}{|c|c|c|c|c|c|}
        \hline
        \multicolumn{2}{|c|}{Setup}                          \\
        \hline
        Worst negative slack (WNS)              & $94.306ns$ \\
        \hline
        Total negative slack (TNS)              & $0.00ns$   \\
        \hline
        Number of failing endpoints             & $0$        \\
        \hline
        Total number of endpoints               & $92$       \\
        \hline
        \multicolumn{2}{|c|}{Hold}                           \\
        \hline
        Worst hold slack (WHS)                  & $0.247ns$  \\
        \hline
        Total hold slack (THS)                  & $0.000ns$  \\
        \hline
        Number of failing endpoints             & $0$        \\
        \hline
        Total number of endpoints               & $92$       \\
        \hline
        \multicolumn{2}{|c|}{Pulse width}                    \\
        \hline
        Worst pulse width slack (WPWS)          & $49.500ns$ \\
        \hline
        Total pulse width negative slack (TPWS) & $0.000$    \\
        \hline
        Number of failing endpoints             & $0$        \\
        \hline
        Total number of endpoints               & $47$       \\
        \hline
    \end{tabular}
    \caption{Timing report}
    \label{table:timingreport}
\end{figure}

\subsection{Simulazioni}

Lo sviluppo del dispositivo in codice VHDL è stato validato tramite i test bench forniti per il progetto. Tali test coprono diversi casi particolari come i casi limite di minima e massima lunghezza del flusso $U$, reset multipli ed elaborazioni successive di diverse sequenze. Oltre a questi, sono stati eseguiti anche diversi test bench generati casualmente tramite uno script il quale prepara diversi flussi $U$, ciascun test bench provato in questa maniera conteneva 1000 flussi differenti che il dispositivo doveva elaborare.

Tutti i test provati sul componente sono andati a buon fine sia in Behavioral, Post Synthesis e Post Implementation Simulation.