Il progetto è stato svolto in diversi step. Inizialmente ho studiato il linguaggio VHDL seguendo varie lezioni tenute dai docenti, integrandole con alcuni video e articoli trovati online. Successivamente sono passato a lavorare su Vivado, sintetizzando diversi esempi svolti a lezione. Questo mi ha permesso di capire il funzionamento del software, come funzionano le simulazioni tramite i test bench e come analizzare le forme d'onda.

Successivamente ho studiato la specifica del progetto e in particolare il funzionamento del convolutore. La comprensione del suo funzionamento è stata veloce principalmente grazie ai grafici autoesplicativi riportati nella specifica. Sono poi passato a studiare l'accesso alla memoria e in questo caso i miei dubbi sono stati chiariti dalla descrizione in VHDL della stessa. Con queste informazioni ho potuto progettare la macchina a stati che inizialmente comprendeva più step perché implementavo direttamente la macchina a stati del convolutore. Una volta resomi conto che potevo ridurre i cicli di clock necessari per computare il flusso in uscita, sono riuscito a ridurre notevolmente il tempo di esecuzione globale.